\paragraph
Annotation efforts undertaken to associate GO terms to maize protein coding genes encompass three different approaches: sequence similarity, protein domain, and CAFA (described in sections 2.1.1, 2.1.2, and 2.1.3, respectively). Methods were implemented using several custom scripts (Bash, Python, and R) available at LINK. These scripts were used to run free, open-source bioinformatics tools (NEED LIST?) to annotate maize protein sequences (designated by gene model name). Each tool used in the pipeline requires a different set of inputs which were obtained from multiple sources (detailed in Supplemental Table SXX).

\paragraph
The pipeline was designed for the maize RefGenv3 genome and 5b+ structural annotations produced for the same genome REF. The genome and protein sequences for Filtered Gene Set (FGS) were downloaded from Gramene Release 42 (Tello-Ruiz et al., 2016). The downloaded protein Fasta file contained sequences for all FGS transcripts (e.g., the gene model X has transcript models X\_T01, X\_T02, and X\_T03). For each gene model only the translated protein sequence derived from the longest transcript was used as input for all three GO annotation methods.
3.1.1 Sequence similarity-based annotation
The sequence similarity based annotation method comprises 3 main steps, namely, 1) sequence similarity calculation, 2) valid hit detection, and 3) inheritance of high-confidence GO annotations. The Basic Local Alignment Search Tool for protein sequences (BLASTP) (Altschul et al., 1990) was used with default parameters to calculate sequence similarity between maize protein sequences and two high-confidence (HC) experimental-based datasets, the Arabidopsis thaliana dataset from The Arabidopsis Information Resource (TAIR) (Berardini et al., 2015) and the “all plants” dataset from UniProt (UniProt Consortium, 2015).  See Table 1 for details. Hits were evaluated using the reciprocal-best-hit (RBH) method from BLASTP results, and high-confidence GO terms were inherited between the RBH of maize and the HC dataset.
3.1.1.1 Maize GO annotation using Arabidopsis dataset
    Arabidopsis has the largest number of high-confidence GO annotations among existing plant model organisms (see Figure 1). A FASTA file of protein sequences along with the cognate GO Annotation File (gaf) were downloaded from TAIR (v.10). The A. thaliana protein file contained predicted protein sequences from all transcripts. As in maize, this file was filtered to retain only the protein sequence derived from longest transcript for each gene. Retained protein sequences from A. thaliana were used to create the A. thaliana BLAST database, and retained maize protein sequences were used to create a maize BLAST database. Maize protein sequences were used to query the A. thaliana BLAST database. Likewise, A. thaliana sequences were used to query the maize BLAST database. Results from both rounds were saved as tab-delimited files, and were processed using a custom R script to detect reciprocal best-hit (RBH) pairs based on BLAST Score. High-confidence GO annotations from A. thaliana genes were inherited to maize genes for each RBH pair to produce the ‘Maize-TAIR GO annotation’ set.
3.1.1.2 Maize GO annotation using All Plants dataset
	The UniProt database includes a collection of protein sequences from a large number of species along with functional annotations. The UniProt-GOA project contains GO annotations for the protein sequences in the UniProt database (Huntley et al., 2015), and the QuickGO interface hosted at the EBI allows easy and bulk downloads of these annotations and sequences (Binns et al., 2009). High-confidence GO annotations and protein sequences for all plants from the Uniprot-GOA database were downloaded using the QuickGO tool hosted at EBI. Data downloaded from UniProt GOA contained a small number of  maize proteins which had manually curated GO annotations. These were removed to create an All Plants protein Fasta file, and a gaf file with high-confidence plant GO annotations. An All Plants protein BLAST database was created. Similar to A. thaliana, two rounds of BLAST searched were performed for the All Plants/maize RBH analysis. Retained maize protein sequences were used to query the All Plants BLAST database. Likewise, retained All Plants sequences were used to query the maize BLAST database. Results from both rounds were saved as tab-delimited files, and were processed using a custom R script to detect RBH pairs based on BLAST Score. High-confidence GO annotations from All Plants genes were inherited to maize genes for each RBH pairs to produce the ‘Maize-Plant GO annotation’ set.
3.1.2 Domain-based annotation
Proteins can be characterized and clustered based on sequence characteristics. Resulting clusters can be used to build protein domain motifs and to detect sites that are likely to be important for protein function. GO terms can be assigned to these domains or clusters based on manual or automated curation. These domains and clusters are generally built using mathematical or statistical models and open source bioinformatics tools are available to search uncharacterized sequences for the presence of known domains or to determine membership in existing protein clusters. Uncharacterized sequences are annotated by detecting known domains or clusters then assigning GO terms from the known domains to the uncharacterized sequences.
Various databases store information on characterized protein domains or families (e.g., Pfam, PIRSF, Superfamily REFS). These databases have tools that could be used to predict whether a input protein sequence has a putative domain or belongs to a certain family, but the number of databases and types of tools needed for a comprehensive analysis is prohibitive. InterPro (REF) is a database that combines information (i.e., signatures) from several protein domain and family databases into a single resource. Moreover, the InterPro project has released InterProScan, software that combines the multiple tools necessary to search for the signatures in a high throughput manner.
InterProScan5 version 5.16-55.0 (default parameters) was used to perform domain based GO annotation of maize protein coding genes. InterProScan5 and necessary databases were downloaded, installed, and configured on the lightning3 HPC cluster (DETAILS) at Iowa State University. The maize filtered protein file was input for direct annotation of GO terms to the proteins. Results were filtered and formatted into a gaf file to produce the ‘Maize-InterProScan GO annotation’ set.
3.1.3 CAFA-based annotation
Functional annotation of nucleic acid and protein sequences has become an important research area with increasing number of genomes that are being sequenced currently. Critical Assessment of protein Function Annotation algorithms (CAFA) is an experiment designed to assess different tools available to predict protein function. This experiment is carried out as a competition, where a common set of proteins are provided as prediction targets. GO annotations are predicted for these targets by various functional annotation tools developed by the research community. Predictions are evaluated using a gold standard test dataset, which is based on experimentally determined annotations for a subset of the prediction targets. The CAFA experiment has enabled the research community working in the area of protein function prediction to come together and evaluate the most recent tools using a common dataset. Comparison of evaluation metrics among function prediction tools from CAFA also enables users to identify top performing tools to annotate their own datasets.
At the beginning of this project, the first iteration of the CAFA challenge (described in the Introduction: CAFA1) had been completed, and the results had been published. The results from the challenge indicated that the new generation of tools performed as well or better than existing methods. To determine their predictive power for maize functional annotation, we used some of the best-performing CAFA annotation methods to predict GO annotations for the maize protein dataset. Based on performance and the availability of code for large-scale protein annotations, we selected three tools: FANN-GO (Clark and Radivojac, 2011), PANNZER (Koskinen et al., 2015), and Argot2 (Falda et al., 2012). FANN-GO had the pre-processing step built into the codebase, while other tools required external tools to be run before the annotation step.
3.1.3.1 FANN-GO based annotation
	FANN-GO is an artificial neural network-based predictor to annotate protein sequences, and uses features based on sequence similarity to predict functions for protein sequences.
Written as MATLAB code, FANN-GO requires MATLAB to execute. The file containing maize filtered protein sequences was imported into MATLAB using a built-in function. The MAIN function from FANN-GO was used to pre-process and annotate maize protein sequences. FANN-GO uses BLASTP to search input sequences against the FANN-GO training sequence dataset (derived from UniProt) and creates input feature vectors. The FANN-GO predictor built from the training dataset is then used to process the input feature vectors and calculate the probability that a particular protein is associated to a particular GO term. These probabilities are represented in a large n x m matrix where rows represent sequences and columns represent GO terms. The large matrix was converted to a gaf file to be used with the AIGO tool for evaluations. A custom Python script was used to filter annotations with probabilities above the optimum threshold to generate the final set of annotations for FANN-GO. This resulted in the creation of a ‘Maize-FANN-GO GO Annotation’ set gaf file.




Running FANN-GO on new protein sequences, consists of two steps. Executing a FANN-GO codebase includes code to run pre-processing steps on raw sequences to create input features, and predict GO annotations.  FANN-GO can predict a score for input sequence for a total of 2132 terms (344 MFO and 1788 BPO).

3.1.3.2 Pannzer based annotations
Pannzer was one of the top performing methods in CAFA1 and was available to download and run locally (Koskinen et al., 2015). Pannzer is written in python, and requires a linux operating system and python to execute the code. A MySQL database is also required to store data to be used by Pannzer. Pre-processing of the input sequences is required to run before executing Pannzer to predict GO annotations. Maize sequences were pre-processed using BLASTP to query the local UniProt protein BLAST database, and the output was saved in XML format as required by Pannzer. The input sequences were split into smaller chunks (200 sequences per file) before they were queried against the UniProt database, to finish the processing within the time limit set by the cluster. This also allowed restarting the job if there were any crashes during the pre-processing or annotation steps. A set of config files as described by pannzer were generated for each output file generated by BLASTP. Pannzer was run using the config files generated in the previous step. A custom python script was used to concatenate the results from Pannzer, and create a gaf file with ‘Maize-Pannzer GO annotation’ set.
3.1.3.3 Argot2 based annotations
	Annotation Retrieval of Gene Ontology Terms (Argot2) was another top performing methods in CAFA1 (Falda et al., 2012). Unfortunately, Argot2 does not have a version that can be downloaded and run locally, but it does have a batch processing tool which can annotate upto 5000 pre-processed input sequences. There are two different pre-processing steps for Argot2 namely, 1) querying the UniProt database for sequence similarity matches to the input sequences, and 2) querying the the Pfam database for putative domains present in the input sequences. The maize sequences were split into multiple fasta files containing a maximum of 5000 sequences. The eight fasta files resulting from the previous step were used to query the UniProt database using BLASTP for matches and the output was saved as tab-delimited files as required by Argot2. Pfam-A and Pfam-B files were downloaded for Pfam release 27.0, and were combined together to create a single local Pfam database. Hmmer version 3.1b1 was downloaded, and was used search the input protein sequences against the local Pfam database. The results from Hmmer search were saved as a table output as required by Argot2. Pre-processing each input fasta resulted in a pair of input files for Argot2, namely BLAST output, and Hmmer output. Each pair of the 8 pairs of pre-processed files were zipped and uploaded to Argot2 batch processing tool to annotate GO terms to maize sequences. Results were downloaded after processing, and combined using a custom script to create a ‘Maize-Argot2 Annotation’ set formatted as a gaf file.
