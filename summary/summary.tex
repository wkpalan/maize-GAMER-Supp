\begin{abstract}

We created a new high-coverage, robust, and reproducible functional annotation of maize protein coding genes based on Gene Ontology (GO) term assignments. Whereas the existing Phytozome and Gramene maize GO annotation sets only cover 41\% and 56\% of maize protein coding genes, respectively, our GO set covers 100\% of the genes. We also compared the quality of our newly-derived annotations with the existing Gramene and Phytozome functional annotation sets by comparing all three to a manually annotated gold standard set of \num{1619} genes where annotations were primarily inferred from direct assay or mutant phenotype. Evaluations against the gold standard indicated that our new annotation set is measurably more accurate than those from Phytozome and Gramene (the $F_1$ score for Phytozome, Gramene, and our new annotation set were 0.094, 0.462, and 0.548, respectively, where higher measures indicate better accuracy). To derive this new high-coverage, high-confidence annotation set we used sequence similarity and protein domain based methods as well as mixed-method pipelines that developed for the Critical Assessment of Function Annotation (CAFA) challenge. Our project to improve maize annotations is called maize-GAMER (GO Annotation Method, Evaluation, and Review) and the newly-derived maize-GAMER annotations are accessible via MaizeGDB  (\href{http://download.maizegdb.org/maize-GAMER}{http://download.maizegdb.org/maize-GAMER}) and CyVerse (B73 RefGen\_v3 5b+ at \href{http://doi.org/10.7946/P2S62P}{doi.org/10.7946/P2S62P} and B73 RefGen\_v4 Zm00001d.2 at \href{http://doi.org/10.7946/P2M925}{doi.org/10.7946/P2M925}).

\end{abstract}
