\section{Introduction}

Maize is an agriculturally important crop species and model organism for genetics and genomics research (reviewed in \cite{lawrence_2004-yE}). Not only is maize historically important for genetics research, along with other model species, significant efforts have been made to transition existing datasets into a more sequence-centric paradigm (reviewed in \cite{sen_2009-Fi}), thus enabling genomics approaches to be brought to bear on both basic research problems and applied breeding (\cite{lawrence_2008-N-}). In 2009 the maize genome's reference sequence was made available to the research community (\cite{schnable_2009-k9}). Since then, much work has gone into improving the utility of the genome sequence to scientists with a focus on sequence annotation.

In practice, making a genome sequence useful involves three basic steps: assembling the genome sequence, assigning gene structures, and annotating functions to genes. The quality of data generated at each step influences downstream inferences, with high-quality sequence, assembly, and gene structure assignments generally resulting in better functional annotations overall. Functional predictions serve as the basis for formulating hypotheses that are subsequently tested in the lab. As such, experimentalists have a great interest in high-quality functional annotation sets that cover all or most of the genes in their species of interest.

The Gene Ontology (GO) is a controlled vocabulary of hierarchically related terms that describe gene product function (\cite{ashburner_2000-2F}). It consists three categories: Biological Process (BP), Cellular Component (CC), and Molecular Function (MF). In the context of GO, functional annotation of a gene consists of the assignment of one or more GO terms from one or more of the GO categories to a given gene or gene model (here we will refer to genes and gene models simply as 'genes' for simplicity).

For individual GO term associations to genes, Evidence Codes (ECs) are assigned to assert how the association of term to gene was made (\cite{harris_2004-KO}). GO evidence codes are aggregated into five general categories: Experimental, Computational Analysis, Curator Statement, Author Statement, and Automatically Assigned. Within each category are 1-10 ECs (see Table \ref{tbl:ev_codes}).

As described in the GO website, (\href{http://geneontology.org/page/guide-go-evidence-codes}{http://geneontology.org/page/guide-go-evidence-codes}) the use of experimental ECs asserts that the assignment results from a physical characterization of the protein's function as described in a publication. Computational approaches are based on \textit{in silico} analyses. One of the simplest and most commonly conducted computational approaches involves matching similar genes between an existing, well-annotated genome and an unannotated genome. Once the matches are assigned, annotations are inferred to genes in the unannotated genome. Such assignments receive the ISS (Inferred from Sequence or Structural Similarity) EC. The ISS EC is also assigned if an uncharacterized sequence contains a characterized domain. In such instances, the presence of the domain itself can be used to predict function for the uncharacterized sequence. For Curator and Author Statements, included EC types are based on judgment by curators and scientists in their expert opinion. As such, they are considered to be reviewed annotation types, though these do include two ECs based on little data: NAS (Non-traceable Author Statement) and ND (No biological Data available). The Automatically Assigned EC type contains only one EC: Inferred from Electronic Annotation (IEA). IEA is unique in that no reviewed analysis of the assignment is required.  Put another way, no curatorial judgment is applied, making it the least supported EC of the group.

Sequence-based approaches to automated functional annotation generally fall into three basic categories: sequence similarity, domain-based methods, and mixed methods. Sequence similarity-based gene matching most often relies on BLAST (e.g., BLAST2GO) followed by limiting the number of accepted matches based on e-value or a reciprocal-best-hit (RBH) strategy  (\cite{conesa_2008-pG, altschul_1990-O4, morenohagelsieb_2008-28}). Domain-based methods score sequences for the presence of well-described protein domain such as those included in Pfam, PANTHER, and ProSite (\cite{finn_2017-cC}). InterProScan is a commonly used domain-based GO annotation pipeline (\cite{jones_2014-tn}). Mixed methods combine sequence similarity, domain-based approaches, and other evidence such as inferred orthology through phylogenetics to assign GO terms systematically (\cite{koskinen_2015-sl, clark_2011--Z,falda_2012-VX}). For more of the latest methods, see \cite{jiang_2016-be}.

For maize, two genome-scale GO annotation sets exist for the B73 reference assembly and gene set (i.e., B73 RefGen\_v3 and 5b+, respectively). These functional annotations are generated by and accessible from the Gramene \sloppy (\href{www.gramene.org}{www.gramene.org}; \cite{telloruiz_2016-qu} and Phytozome \href{phytozome.jgi.doe.gov}{phytozome.jgi.doe.gov}; \cite{goodstein_2012-AM}) projects and websites, respectively. Gramene annotations are based on the Ensembl annotation pipeline
(\href{http://ensemblgenomes.org/info/data/cross\_references}{http://ensemblgenomes.org/info/data/cross\_references}), which is a mixed-method approach. The primary sources of the Ensembl annotations are from UniProtKB, community-based annotations from MaizeGDB (\cite{andorf_2016-wk}), InterPro2GO, and projections from orthologs inferred from phylogenetic analyses. Phytozome has a two-step process for GO annotation.  First, Pfam domains are assigned to proteins. Second, GO annotations are determined based on the Pfam2GO mapping (\cite{hunter_2009-6D}.)

Given the wealth of functional descriptions derived from mutational analyses, many researchers rely on the available maize GO-based functional annotations from large-scale, high-profile community resources like Gramene and Phytozome for formulating experimental hypotheses, and also as input datasets to transitively annotate predicted functions to newly sequenced grass species and crop genomes (e.g. \cite{hirsch_2016-Gs}). However, if we compare the EC types for GO assignments between the model species \emph{Arabidopsis thaliana} and the Gramene and Phytozome functional annotations of the maize reference line B73, it is clear that the evidence supporting GO term assignments for these maize datasets is comparatively lacking (see Figure \ref{fig:num_annots}. Both the Gramene and Phytozome maize annotations have few annotations beyond those Inferred from Electronic Annotation (IEA). This situation is not intuitive to researchers given that maize has a wealth of functional descriptions in the literature.

Exacerbating this problem, transfers of predicted function often are based on sequence similarity alone with no restriction of input data to associations based on well-documented EC types. What's more, although mixed method pipelines like the Ensemble COMPARA pipeline used by Gramene and the Phytozome Pfam2GO (\cite{herrero_2016-k1,goodstein_2012-AM}) mappings may seem reproducible in principle given that they are based on the use of specific systems and software, details including input files and parameters often are unavailable or incomplete, making it impossible for research groups outside the group that generated those annotation resources to reproduce the annotation sets. In addition, because many computational pipelines inherit functional annotations that were also purely computationally derived, a single errant annotation can be propagated to many genomes (\cite{andorf_2007-5F}), making it mistakenly appear that many genomes agree on the errant function.  For these reasons, existing computational functional annotations of maize (and many other plant genomes) should be approached with skepticism.

Given these issues with the maize functional annotation, we endeavored to create an improved annotation set. This task requires both application of robust and reproducible methods and a gold standard set of maize GO annotations to compare generated result sets to each other as well as to the Gramene and Phytozome maize functional annotations. One small dataset of well-curated GO-based functional annotations does exist for maize. It was initially created by curators at MaizeGDB for the purpose of enriching the MaizeCyc metabolic pathway database (\cite{monaco_2013-GY}) and expanded through manual literature curation. This dataset constitutes \num{1621} genes and \num{2002} GO terms.

We annotated the maize B73 RefGen\_v3 annotation set 5b+. We used only experimentally-based annotations by filtering out GO assignments with IEA, NAS, and ND ECs from the input data, assigned GO terms using multiple input datasets, and compared the performance of sequence similarity, domain-based, and mixed methods based on how well the methods predicted function for genes included in the MaizeGDB gold standard dataset. For mixed methods, we used pipelines developed for the Critical Assessment of Functional Annotation (CAFA) challenge, a competition designed to evaluate the latest computational functional annotation methods and to promote improvement of  methods for functional annotation (\cite{radivojac_2013-YN, jiang_2016-be}). Groups competing in the CAFA challenge create tools that are are applied to a set of specified target sequences. GO assignments are subsequently evaluated based on accumulation of functional data in the literature for the target sequence set. Some CAFA tools use pre-processing steps combined with a number of different computational and statistical approaches to reduce the number of false positive and false negative annotations (\cite{clark_2011--Z,koskinen_2015-sl,falda_2012-VX}). Some mixed-method pipelines performed better on average than other methods in the first iteration of the CAFA competition (\cite{radivojac_2013-YN}), indicating that the use of mixed-method pipelines for large scale GO annotations could potentially improve the overall quality of produced annotation sets.

 The pipelines and suite of tools we developed is called GAMER: \underline{G}O \underline{A}nnotation \underline{M}ethod, \underline{E}valuation, and \underline{R}eview. We compared GAMER annotations to annotations based on sequence-similarity, domain, and some CAFA mixed-methods. Next we combined GAMER outputs to generate aggregate maize-GAMER GO annotations and compared it to the existing Phytozome and Gramene GO annotations based on the F-measure statistic. The GAMER annotations had three major advantages compared to the Gramene and Phytozome annotations: (1) an increased number of maize genes annotated with GO terms; (2) more than twice the number of annotations (GO terms assigned) for maize protein coding genes; (3) similar or better quality scores relative to  existing annotations sets based on $hF_1$ score. The B73 RefGen\_v3 5b+ maize-GAMER functional annotation dataset described here is accessible via MaizeGDB (\href{http://download.maizegdb.org/maize-GAMER}{http://download.maizegdb.org/maize-GAMER}) and CyVerse (\href{http://doi.org/10.7946/P2S62P}{doi.org/10.7946/P2S62P}). Scripts used to generate the annotation are available via GitHub at \href{https://github.com/Dill-PICL/maize-GAMER}{https://github.com/Dill-PICL/maize-GAMER}. \textit{(Note for reviewers: the annotations will also be made available via each MaizeGDB gene model page upon publication.)}