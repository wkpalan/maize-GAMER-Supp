\begin{table}[h]
  \centering
  \begin{tabularx}{\textwidth}{lXrrr}
    \hline
    \bf{Metric} & \textbf {Ontology} & \bf{\# of Genes}  & \textbf {TAIR-ALL} & \textbf {TAIR-HC} \\
    \hline
    \multirow{4}{*}{Average hF-score}
    & BP&152 & 0.1802 & 0.1700 \\
    & MF&60 & 0.2967 & 0.3182 \\
    & CC&1537 & 0.4286 & 0.4279 \\
    & Overall&1749 & 0.4025 & 0.4017 \\
    \hline
    Nb. of Annotations & All &  & 84,569 & 61,530 \\
    \hline
  \end{tabularx}

  \raggedright
  \caption{Comparison of Average hierarchical F-scores between TAIR-ALL and TAIR-HC datasets}\label{table:tair-rbh}
  \textbf{TAIR-ALL} represents the dataset which was generated by inheriting GO terms without filtering out non high-confidence TAIR GO annotations.
  \textbf{TAIR-HC} represents the dataset which was generated by inheriting GO terms after filtering out non high-confidence TAIR GO annotations.
  \textbf{\# of Genes} represent the count of  genes that have at least one annotation in a specific ontology in the gold standard dataset.

  Average hierarchical F-score were calculated by comparing the annotations from both datasets to the gold standard dataset (see methods for details). The values were calculated for each ontology separately for all the genes that had gold standard dataset. Overall scores were calculated averaging across the entire gold standard dataset for all genes. Number of resulting annotations are obtained by counting all annotations after inheriting annotation from Arabidopsis.
\end{table}
